% Accents
\usepackage[utf8]{inputenc}
\usepackage[T1]{fontenc}
\usepackage[portuguese]{babel}
\usepackage{url}
\usepackage{indentfirst}

% Mathematics
\usepackage{amsfonts}
\def \grad #1 {\nabla #1}
\def \div #1 {\nabla \cdot #1}
\def \curl #1 {\nabla \times #1}
\def \transpose #1 {{#1}^t}
\def \restriction #1#2 {\left.#1\right|_{#2}}
\def \deriv #1#2 {\frac{d #1}{d #2}}
\def \pderiv #1#2 {\frac{\partial #1}{\partial #2}}
\usepackage{gensymb}

% Algorithm
\usepackage{algorithm}
\usepackage{algpseudocode}
\usepackage{pifont}

%Enumerate
\usepackage{enumitem}

% Units
%
% Example:
%
%	\SI{"NUMBER"}{"UNITS"}
%
\usepackage{siunitx}

% Bibliography
\usepackage[square]{natbib}

%% Figure
%
% Exemple:
%
%	\fig[hd_tits][.7]{Main caption}{Further explanation...}
%   \ref{fig:hd_tits}
%
\def \fig [#1][#2]#3#4 {
  \begin{figure}[H]
    \centering
      \includegraphics[width=#2\textwidth]{images/#1}
      \caption{#3}{\footnotesize{#4}}
      \label{fig:#1}
  \end{figure}
}

%% Subfigures
%
% Exemple:
%
%	\subfigs[figure_with_subfigures]{
%		\subfig[sub_one][.3]{First}
%		\subfig[sub_two][.3]{Second}
%		\subfig[sub_thr][.3]{Third}
%	}{Main caption}{Further explanation...}
%
\usepackage{graphicx}
\usepackage{caption}
\usepackage{subcaption}
\def \subfigs [#1]#2#3#4 {
  \begin{figure}[!ht]
    \centering
      #2
      \caption{#3}{\footnotesize{#4}}
      \label{fig:#1}
  \end{figure}
}
\def \subfig [#1][#2]#3 {
  \begin{subfigure}[b]{#2\textwidth}
    \includegraphics[width=\textwidth]{images/#1}
    \caption{#3}
    \label{fig:#1}
  \end{subfigure}
}

%% Aqui
\usepackage{xargs}
\usepackage[pdftex,dvipsnames]{xcolor}
\usepackage[colorinlistoftodos,prependcaption,textsize=tiny]{todonotes}
\newcommandx{\barradomal}{\todo[inline]{\newline {\huge A FAMOSA BARRA DO MAL} \newline}}

% RMR50
\def \RMR {{\emph{\textsf{RMR50~}}}}
